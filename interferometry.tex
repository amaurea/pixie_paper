\documentclass{article}
\usepackage{amsmath}
\usepackage[margin=22mm]{geometry}
\newcommand{\rtwo}{\frac{1}{\sqrt{2}}}
\newcommand{\rtd}{\frac{\delta^+}{\sqrt{2}}}
\newcommand{\rtc}{\frac{\delta^-}{\sqrt{2}}}
\newcommand{\I}{\tilde I}
\newcommand{\Q}{\tilde Q}
\newcommand{\U}{\tilde U}
\newcommand{\V}{\tilde V}
\newcommand{\J}{{\tilde E}}
\renewcommand{\Re}{\operatorname{Re}}
\renewcommand{\Im}{\operatorname{Im}}

\begin{document}
PIXIE has two barrels, A and B. We can expand the electric field $\vec E^A(t)$,
$\vec E^B(t)$ that enters these barrels in terms of Jones vectors as
\begin{align}
	\vec E^A(t) &= \Re \int_0^\infty d\omega \Big(\J^A_x(\omega)\vec e_x +
		\J^A_y(\omega)\vec e_y\Big) e^{i(kz-\omega t)} \notag \\
	\vec E^B(t) &= \Re \int_0^\infty d\omega \Big(\J^B_x(\omega)\vec e_x +
		\J^B_y(\omega)\vec e_y\Big) e^{i(kz-\omega t)}
\end{align}
where $\omega$ is the angular frequency of the radiaton and $\vec \J^A(\omega)$
and $\vec \J^B(\omega)$ are (complex) Jones vectors at that
angular frequency.

After entering the barrels the light encounters polarizer A, which lets through
vertical polarization and reflects horizontal\footnote{
	Before this it encounters the primary mirror, folding flats, secondary mirror,
	and transfer mirror 1, but these lead to the same phase shifts on both
	the A and B side optical paths, so they can be neglected.}. Afer this,
the Jones vectors in left (A) and right (B) shafts are
\begin{align}
	\vec \J^{A1} &= \J^A_x \vec e_x + \J^B_y \vec e_y &
	\vec \J^{B1} &= \J^B_x \vec e_x + \J^A_y \vec e_y
\end{align}
After passing through the diagonal polarizer B, we have
\begin{align}
	\vec \J^{A2} &= \rtwo [\J^A_x+\J^B_y]\vec e_a + \rtwo[-\J^B_x+\J^A_y]\vec e_b \notag \\
	\vec \J^{B2} &= \rtwo [\J^B_x+\J^A_y]\vec e_a + \rtwo[-\J^A_x+\J^B_y]\vec e_b
\end{align}
where $\vec e_a \equiv \rtwo [\vec e_x + \vec e_y]$ and $\vec e_b
\equiv \rtwo [-\vec e_x + \vec e_y]$. The dihedral mirror then
imparts a path length difference between the two sides, advancing
A by $\frac12\Delta t$ and retarding B by $\frac12\Delta t$, which
is achieved by multiplying A by $\delta^+ = e^{-\frac12\omega\Delta t}$
and B by $\delta^- = e^{i\omega\Delta t}$:
\begin{align}
	\vec \J^{A3} &= \rtd [\J^A_x+\J^B_y]\vec e_a + \rtd[-\J^B_x+\J^A_y]\vec e_b \notag \\
	\vec \J^{B3} &= \rtc [\J^B_x+\J^A_y]\vec e_a + \rtc[-\J^A_x+\J^B_y]\vec e_b
\end{align}
Polarizer C is also diagonal.
\begin{align}
	\vec \J^{A4} &= \rtd [\J^A_x+\J^B_y]\vec e_a + \rtc [-\J^A_x+\J^B_y]\vec e_b \notag \\
	\vec \J^{B4} &= \rtc [\J^B_x+\J^A_y]\vec e_a + \rtd [-\J^B_x+\J^A_y]\vec e_b
\end{align}
And the final polarizer D is vertical. The output of this enters
the left (L) and right (R) feedhorns.
\begin{align}
	\vec \J^L = \vec \J^{A5} &= \frac12 \Big[\delta^+(\J^A_x+\J^B_y)+\delta^-(\J^A_x-\J^B_y)\Big]\vec e_x
		+ \frac12 \Big[\delta^-(\J^B_x+\J^A_y)+\delta^+(-\J^B_x+\J^A_y)\Big]\vec e_y \notag \\
		&= \big[\J^A_x\cos(\omega\Delta t/2) + i\J^B_y\sin(\omega\Delta t/2)\big]\vec e_x
		+  \big[\J^A_y\cos(\omega\Delta t/2) - i\J^B_x\sin(\omega\Delta t/2)\big]\vec e_y \\
	\vec \J^R = \vec \J^{B5} &= \frac12 \Big[\delta^-(\J^B_x+\J^A_y)+\delta^+(\J^B_x-\J^A_y)\Big]\vec e_x
		+ \frac12 \Big[\delta^+(\J^A_x+\J^B_y)+\delta^-(-\J^A_x+\J^B_y)\Big]\vec e_y \notag \\
		&= \big[\J^B_x\cos(\omega\Delta t/2) - i\J^A_y\sin(\omega\Delta t/2)\big]\vec e_x
		+  \big[\J^B_y\cos(\omega\Delta t/2) + i\J^A_x\sin(\omega\Delta t/2)\big]\vec e_y
\end{align}
After passing through all this, the light enters the feedhorns and
hits the detectors. The power deposited here can be decomposed into
Stokes parameters
\begin{align}
	\I &= \langle |\J_x|^2\rangle + \langle|\J_y|^2\rangle &
	\Q &= \langle |\J_x|^2\rangle - \langle|\J_y|^2\rangle &
	\U &= 2\Re\langle \J_x\J_y^*\rangle &
	\V &= -2\Im\langle \J_x\J_y^*\rangle
\end{align}
so we need to evaluate $P_{xx} = \langle |\J_x|^2\rangle$,
$P_{yy} = \langle |\J_y|^2\rangle$ and $P_{xy} = \langle \J_x\J_y^*\rangle$.
For the left horn we get
\begin{align}
	P^L_{xx} &= \langle \J^L_x\J^{L*}_x \rangle \notag \\
		&= \frac12\big[1+\cos(\omega\Delta t)\big]\langle\J^A_x\J^{A*}_x\rangle
		+ \frac12\big[1-\cos(\omega\Delta t)\big]\langle\J^B_y\J^{B*}_y\rangle
		- \frac{i}2\langle\J^A_x\J^{B*}_y - \J^{A*}_x\J^B_y\rangle\sin(\omega\Delta t) \notag \\
		&= \frac14\big[\I^A+\I^B+\Q^A-\Q^B + (\I^A-\I^B+\Q^A+\Q^B)\cos(\omega\Delta t) +
		4\Im(\J^A_x\J^{B*}_y)\sin(\omega\Delta t) \big] \\
	P^L_{yy} &= \frac14\big[\I^B+\I^A+\Q^B-\Q^A - (\I^B-\I^A+\Q^B+\Q^A)\cos(\omega\Delta t) +
		4\Im\langle\J^B_x\J^{A*}_y\rangle\sin(\omega\Delta t) \big ]\\
	P^L_{xy} &= \frac14\big[\U^A-\U^B - i\V^A-i\V^B + (\U^A+\U^B -i\V^A+i\V^B)\cos(\omega\Delta t)
		-2i\langle\J^A_x\J^{B*}_x+\J^B_y\J^{A*}_y\rangle\sin(\omega\Delta t)
\end{align}
Hence
\begin{align}
	\I^L &= \frac12\big[\I^A+\I^B+(\I^A-\I^B)\cos(\omega\Delta t)
		+2\Im\langle\J^A_x\J^{B*}_y+\J^B_x\J^{A*}_y\rangle\sin(\omega\Delta t) \big] \\
	\Q^L &= \frac12\big[\Q^A-\Q^B+(\Q^A+\Q^B)\cos(\omega\Delta t)
		+2\Im\langle\J^A_x\J^{B*}_y-\J^B_x\J^{A*}_y\rangle\sin(\omega\Delta t) \big] \\
	\U^L &= \frac12\big[\U^A-\U^B+(\U^A+\U^B)\cos(\omega\Delta t)
		+2\Im\langle\J^A_x\J^{B*}_x+\J^B_y\J^{A*}_y\rangle\sin(\omega\Delta t) \big] \\
	\V^L &= \frac12\big[\V^A+\V^B+(\V^A-\V^B)\cos(\omega\Delta t)
		+2\Re\langle\J^A_x\J^{B*}_x+\J^B_y\J^{A*}_y\rangle\sin(\omega\Delta t) \big]
\end{align}
The value of the barrel cross-terms depends on whether PIXIE is in single
or double barrel mode.

\paragraph{Single barrel mode}
In single barrel mode only one barrel is exposed to the sky; the other one
observes a static calibrator object. The light entering the two barrels is
therefore uncorrelated, and all the cross-terms disappear.
\begin{align}
	\I^L &= \frac12\big[\I^A+\I^B+(\I^A-\I^B)\cos(\omega\Delta t)\big] &
	\Q^L &= \frac12\big[\Q^A-\Q^B+(\Q^A+\Q^B)\cos(\omega\Delta t)\big] \\
	\V^L &= \frac12\big[\V^A+\V^B+(\V^A-\V^B)\cos(\omega\Delta t)\big] &
	\U^L &= \frac12\big[\U^A-\U^B+(\U^A+\U^B)\cos(\omega\Delta t)\big]
\end{align}

\paragraph{Double barrel mode}
In double barrel mode the two barrels are both coaligned exposed to the sky,
so they observe the same wavefront entering. As PIXIE's angular resolution is
not infinite it is sensitive to wavefronts that are off-axis by a few degrees.
Light arriving from direction $\hat n$ will hit Barrel B with a delay
of $\tau = -\hat n \cdot \vec b / c$ where $\vec b$ is the distance vector from
barrel A to barrel B, and $c$ is the speed of light (see~\cite{pixie_array}).
[Won't the different angles the mirrors are hit with in the two optical paths
matter here too? I guess that can be absorbed into $\tau$]
So in this case $\vec \J^B = \gamma \vec\J^A$ with $\gamma = e^{-i\omega\tau}$.
\begin{align}
	\I^L &= \I^A-\V^A\cos(\omega\tau)\sin(\omega\Delta t) &
	\Q^L &= \Q^A\cos(\omega\Delta t)+\U^A\sin(\omega\tau)\sin(\omega\Delta t) \\
	\V^L &= \V^A+I^A\cos(\omega\tau)\sin(\omega\Delta t) &
	\U^L &= \U^A\cos(\omega\Delta t)+\Q^A\sin(\omega\tau)\sin(\omega\Delta t)
\end{align}







Using the expression above, we can relate the
Stokes parameters incident on the detectors to the Stokes parameters
of the radiation that enters the barrels. The parameters are defined
as $I \equiv \langle E_x^2\rangle + \langle E_y^2\rangle$,
$Q \equiv \langle E_x^2\rangle - \langle E_y^2\rangle$,
$U \equiv 2\langle E_xE_y\rangle$, so we need to compute $\langle E_x^2\rangle$,
$\langle E_y^2\rangle$ and $\langle E_xE_y\rangle$. For the A path, the first of these is

\begin{align}
	P^L_x &= \langle {E^{A5}_x}^2\rangle \notag \\
		&= \frac14\big[
			\langle {E^{A+}_x}^2\rangle
			+ \langle {E^{B+}_y}^2\rangle
			+ \langle {E^{A-}_x}^2\rangle
			+ \langle {E^{B-}_y}^2\rangle
			+ 2\langle E^{A+}_xE^{B+}_y\rangle
			+ 2\langle E^{A+}_xE^{A-}_x\rangle \notag \\
			&- 2\langle E^{A+}_xE^{B-}_y\rangle
			+ 2\langle E^{B+}_yE^{A-}_x\rangle
			- 2\langle E^{B+}_yE^{B-}_y\rangle
			- 2\langle E^{A-}_xE^{B-}_y\rangle
			\big]
\end{align}
If we assume that the sky does not change over time-scales
comparable to the period of the light, then $\langle {E^{A+}}^2\rangle
= \langle {E^{A-}}^2\rangle = \langle {E^A}^2\rangle = \frac12(I^A+Q^A)$,
and similarly for B. The cross-terms $\langle E^{A+}_xE^{B-}_y\rangle
-\langle E^{A-}_xE^{B+}_y\rangle$ can be ignored in both of Pixie's modes of
observation. In single barrel mode the A and B signals are uncorrelated,
making both individually zero. In double barrel mode the A and B signals are
the same, so $\langle E^{A+}_xE^{B-}_y\rangle
-\langle E^{A-}_xE^{B+}_y\rangle = \langle E^{A+}_xE^{A-}_y\rangle
-\langle E^{A-}_xE^{A+}_y\rangle = \frac12$ TODO fix this with
Jones angles form the beginning

By similar logic, we see that $\langle E^{A+}_xE^{B-}_y\rangle = \langle E^{A-}_xE^{B+}_y\rangle$ (actually, I'm not sure this is true). Using this, we get
\begin{align}
	P^L_x &= \frac14 [I^A+I^B+Q^A-Q^B + I^{A\Delta}+Q^{A\Delta}-I^{B\Delta}+Q^{B\Delta}]
\end{align}
where $I^\Delta \equiv \langle E^+_xE^-_x\rangle +\langle E^+_yE^-_y\rangle$, and similarly for the
other stokes parameters. We can write the result more symmetrically by
defining $[\textrm{expr}]_\Delta$ to mean the expression evaluated with
a delay of $\Delta$. With this, we get
\begin{align}
	P^L_x &= \frac14 [I^A+I^B+Q^A-Q^B]_0 + \frac14[I^A-I^B+Q^A+Q^B]_\Delta \\
	P^L_y &= \frac14 [I^A+I^B-Q^A+Q^B]_0 + \frac14[I^A-I^B-Q^A-Q^B]_\Delta \\
	P^L_{xy} &= \frac14 [U^A-U^B]_0 + \frac14[U^A+U^B]_\Delta
\end{align}
which results in horn stoks parameters of
\begin{align}
	I^L &= \frac12 [I^A+I^B]_0 + \frac12[I^A-I^B]_\Delta \\
	Q^L &= \frac12 [Q^A-Q^B]_0 + \frac12[Q^A+Q^B]_\Delta \\
	U^L &= \frac12 [U^A-U^B]_0 + \frac12[U^A+U^B]_\Delta
\end{align}
The other horn follows by symmetry.
\begin{align}
	I^R &= \frac12 [I^B+I^A]_0 + \frac12[I^B-I^A]_\Delta \\
	Q^R &= \frac12 [Q^B-Q^A]_0 + \frac12[Q^B+Q^A]_\Delta \\
	U^R &= \frac12 [U^B-U^A]_0 + \frac12[U^B+U^A]_\Delta
\end{align}

\end{document}
