\documentclass{article}
\usepackage{amsmath}
\newcommand{\rtwo}{\frac{1}{\sqrt{2}}}

\begin{document}
This derivation ignores sign changes in the y polarization when reflecting
off mirrors. This appears to cancel in the end. But it's probably best to just
point people at the derivation in the pixie paper appendix.

Radiation with electrical field $\vec E^A = E^A_x \vec e_x + E^A_y \vec e_y$
enters barrel A, and similarly for barrel B. After passing through polarizer
A, which lets through vertical polarization and reflects horizontal, the
electrical fields in the left and right shafts are
\begin{align}
	\vec E^{A1} &= E^A_x \vec e_x + E^B_y \vec e_y
& \vec E^{B1} &= E^B_x \vec e_x + E^A_y \vec e_y
\end{align}
After passing through the diagonal polarizer B, we have
\begin{align}
	\vec E^{A2} &= \rtwo [E^A_x+E^B_y]\vec e_a + \rtwo[-E^B_x+E^A_y]\vec e_b \notag \\
	\vec E^{B2} &= \rtwo [E^B_x+E^A_y]\vec e_a + \rtwo[-E^A_x+E^B_y]\vec e_b
\end{align}
where $\vec e_a \equiv \rtwo [\vec e_x + \vec e_y]$ and $\vec e_b
\equiv \rtwo [-\vec e_x + \vec e_y]$. The dihedral mirror then
imparts a relative phase delay $\Delta$ for the left and right paths.
This advances one side (the A side for positive $\Delta$) and
retards the other. We indicate this with + and - tags.
\begin{align}
	\vec E^{A3} &= \rtwo [E^{A+}_x+E^{B+}_y]\vec e_a + \rtwo[-E^{B+}_x+E^{A+}_y]\vec e_b \notag \\
	\vec E^{B3} &= \rtwo [E^{B-}_x+E^{A-}_y]\vec e_a + \rtwo[-E^{A-}_x+E^{B-}_y]\vec e_b
\end{align}
Polarizer C is also diagonal.
\begin{align}
	\vec E^{A4} &= \rtwo [E^{A+}_x+E^{B+}_y]\vec e_a + \rtwo[-E^{A-}_x+E^{B-}_y]\vec e_b \notag \\
	\vec E^{B4} &= \rtwo [E^{B-}_x+E^{A-}_y]\vec e_a + \rtwo[-E^{B+}_x+E^{A+}_y]\vec e_b
\end{align}
And the final polarizer D is vertical. The output of this enters
the left (L) and right (R) feedhorns.
\begin{align}
	\vec E^L = \vec E^{A5} &= \frac12 [E^{A+}_x+E^{B+}_y+E^{A-}_x-E^{B-}_y]\vec e_x \notag \\
		&+ \frac12 [E^{B-}_x+E^{A-}_y-E^{B+}_x+E^{A+}_y]\vec e_y \notag \\
	\vec E^R = \vec E^{B5} &= \frac12 [E^{B-}_x+E^{A-}_y+E^{B+}_x-E^{A+}_y]\vec e_x \notag \\
		&+ \frac12 [E^{A+}_x+E^{B+}_y-E^{A-}_x+E^{B-}_y]\vec e_y
\end{align}
After passing through all this, the light enters the feedhorns and
hits the detectors. Using the expression above, we can relate the
Stokes parameters incident on the detectors to the Stokes parameters
of the radiation that enters the barrels. The parameters are defined
as $I \equiv \langle E_x^2\rangle + \langle E_y^2\rangle$,
$Q \equiv \langle E_x^2\rangle - \langle E_y^2\rangle$,
$U \equiv 2\langle E_xE_y\rangle$, so we need to compute $\langle E_x^2\rangle$,
$\langle E_y^2\rangle$ and $\langle E_xE_y\rangle$. For the A path, the first of these is

\begin{align}
	P^L_x &= \langle {E^{A5}_x}^2\rangle \notag \\
		&= \frac14\big[
			\langle {E^{A+}_x}^2\rangle
			+ \langle {E^{B+}_y}^2\rangle
			+ \langle {E^{A-}_x}^2\rangle
			+ \langle {E^{B-}_y}^2\rangle
			+ 2\langle E^{A+}_xE^{B+}_y\rangle
			+ 2\langle E^{A+}_xE^{A-}_x\rangle \notag \\
			&- 2\langle E^{A+}_xE^{B-}_y\rangle
			+ 2\langle E^{B+}_yE^{A-}_x\rangle
			- 2\langle E^{B+}_yE^{B-}_y\rangle
			- 2\langle E^{A-}_xE^{B-}_y\rangle
			\big]
\end{align}
If we assume that the sky does not change over time-scales
comparable to the period of the light, then $\langle {E^{A+}}^2\rangle
= \langle {E^{A-}}^2\rangle = \langle {E^A}^2\rangle = \frac12(I^A+Q^A)$,
and similarly for B. By similar logic, we see that $\langle E^{A+}_xE^{B-}_y\rangle = \langle E^{A-}_xE^{B+}_y\rangle$. Using this, we get
\begin{align}
	P^L_x &= \frac14 [I^A+I^B+Q^A-Q^B + I^{A\Delta}+Q^{A\Delta}-I^{B\Delta}+Q^{B\Delta}]
\end{align}
where $I^\Delta \equiv \langle E^+_xE^-_x\rangle$, and similarly for the
other stokes parameters. We can write the result more symmetrically by
defining $[\textrm{expr}]_\Delta$ to mean the expression evaluated with
a delay of $\Delta$. With this, we get
\begin{align}
	P^L_x &= \frac14 [I^A+I^B+Q^A-Q^B]_0 + \frac14[I^A-I^B+Q^A+Q^B]_\Delta \\
	P^L_y &= \frac14 [I^A+I^B-Q^A+Q^B]_0 + \frac14[I^A-I^B-Q^A-Q^B]_\Delta \\
	P^L_{xy} &= \frac14 [U^A-U^B]_0 + \frac14[U^A+U^B]_\Delta
\end{align}
which results in horn stoks parameters of
\begin{align}
	I^L &= \frac12 [I^A+I^B]_0 + \frac12[I^A-I^B]_\Delta \\
	Q^L &= \frac12 [Q^A-Q^B]_0 + \frac12[Q^A+Q^B]_\Delta \\
	U^L &= \frac12 [U^A-U^B]_0 + \frac12[U^A+U^B]_\Delta
\end{align}
The other horn follows by symmetry.
\begin{align}
	I^R &= \frac12 [I^B+I^A]_0 + \frac12[I^B-I^A]_\Delta \\
	Q^R &= \frac12 [Q^B-Q^A]_0 + \frac12[Q^B+Q^A]_\Delta \\
	U^R &= \frac12 [U^B-U^A]_0 + \frac12[U^B+U^A]_\Delta
\end{align}

\end{document}
