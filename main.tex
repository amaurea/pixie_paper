\documentclass{article}
\usepackage{amsmath,gensymb}
\usepackage[margin=22mm]{geometry}
\newcommand{\rtwo}{\frac{1}{\sqrt{2}}}
\newcommand{\rtd}{\frac{\delta^+}{\sqrt{2}}}
\newcommand{\rtc}{\frac{\delta^-}{\sqrt{2}}}
\newcommand{\I}{\tilde I}
\newcommand{\Q}{\tilde Q}
\newcommand{\U}{\tilde U}
\newcommand{\V}{\tilde V}
\newcommand{\J}{{\tilde E}}
\renewcommand{\Re}{\operatorname{Re}}
\renewcommand{\Im}{\operatorname{Im}}

\begin{document}

\section{The PIXIE fourier transform spectrometer}
PIXIE has two barrels, A and B. We can expand the electric field $\vec E^A(t)$,
$\vec E^B(t)$ that enters these barrels in terms of Jones vectors as
\begin{align}
	\vec E^A(t) &= \Re \int_0^\infty d\omega \Big(\J^A_x(\omega)\vec e_x +
		\J^A_y(\omega)\vec e_y\Big) e^{i(kz-\omega t)} \notag \\
	\vec E^B(t) &= \Re \int_0^\infty d\omega \Big(\J^B_x(\omega)\vec e_x +
		\J^B_y(\omega)\vec e_y\Big) e^{i(kz-\omega t)}
\end{align}
where $\omega$ is the angular frequency of the radiaton and $\vec \J^A(\omega)$
and $\vec \J^B(\omega)$ are (complex) Jones vectors at that
angular frequency.

After entering the barrels the light encounters polarizer A, which lets through
vertical polarization and reflects horizontal\footnote{
	Before this it encounters the primary mirror, folding flats, secondary mirror,
	and transfer mirror 1, but these lead to the same phase shifts on both
	the A and B side optical paths, so they can be neglected.}. Afer this,
the Jones vectors in left (A) and right (B) shafts are
\begin{align}
	\vec \J^{A1} &= \J^A_x \vec e_x + \J^B_y \vec e_y &
	\vec \J^{B1} &= \J^B_x \vec e_x + \J^A_y \vec e_y
\end{align}
After passing through the diagonal polarizer B, we have
\begin{align}
	\vec \J^{A2} &= \rtwo [\J^A_x+\J^B_y]\vec e_a + \rtwo[-\J^B_x+\J^A_y]\vec e_b &
	\vec \J^{B2} &= \rtwo [\J^B_x+\J^A_y]\vec e_a + \rtwo[-\J^A_x+\J^B_y]\vec e_b
\end{align}
where $\vec e_a \equiv \rtwo [\vec e_x + \vec e_y]$ and $\vec e_b
\equiv \rtwo [-\vec e_x + \vec e_y]$. The dihedral mirror then
imparts a path length difference between the two sides, advancing
A by $\frac12\Delta t$ and retarding B by $\frac12\Delta t$, which
is achieved by multiplying A by $\delta^+ = e^{-\frac12\omega\Delta t}$
and B by $\delta^- = e^{i\omega\Delta t}$:
\begin{align}
	\vec \J^{A3} &= \rtd [\J^A_x+\J^B_y]\vec e_a + \rtd[-\J^B_x+\J^A_y]\vec e_b &
	\vec \J^{B3} &= \rtc [\J^B_x+\J^A_y]\vec e_a + \rtc[-\J^A_x+\J^B_y]\vec e_b
\end{align}
Polarizer C is also diagonal.
\begin{align}
	\vec \J^{A4} &= \rtd [\J^A_x+\J^B_y]\vec e_a + \rtc [-\J^A_x+\J^B_y]\vec e_b &
	\vec \J^{B4} &= \rtc [\J^B_x+\J^A_y]\vec e_a + \rtd [-\J^B_x+\J^A_y]\vec e_b
\end{align}
And the final polarizer D is vertical. The output of this enters
the left (L) and right (R) feedhorns.
\begin{align}
	\vec \J^L = \vec \J^{A5} &= \frac12 \Big[\delta^+(\J^A_x+\J^B_y)+\delta^-(\J^A_x-\J^B_y)\Big]\vec e_x
		+ \frac12 \Big[\delta^-(\J^B_x+\J^A_y)+\delta^+(-\J^B_x+\J^A_y)\Big]\vec e_y \notag \\
		&= \big[\J^A_x\cos(\omega\Delta t/2) - i\J^B_y\sin(\omega\Delta t/2)\big]\vec e_x
		+  \big[\J^A_y\cos(\omega\Delta t/2) + i\J^B_x\sin(\omega\Delta t/2)\big]\vec e_y \\
	\vec \J^R = \vec \J^{B5} &= \frac12 \Big[\delta^-(\J^B_x+\J^A_y)+\delta^+(\J^B_x-\J^A_y)\Big]\vec e_x
		+ \frac12 \Big[\delta^+(\J^A_x+\J^B_y)+\delta^-(-\J^A_x+\J^B_y)\Big]\vec e_y \notag \\
		&= \big[\J^B_x\cos(\omega\Delta t/2) + i\J^A_y\sin(\omega\Delta t/2)\big]\vec e_x
		+  \big[\J^B_y\cos(\omega\Delta t/2) - i\J^A_x\sin(\omega\Delta t/2)\big]\vec e_y
\end{align}

\subsection{Stokes parameters}
After passing through all this, the light enters the feedhorns and
hits the detectors. The power deposited here can be decomposed into
Stokes parameters\footnote{The quantities with tildes are for a single plane wave.
The full Stokes parameters are obtained by integrating these. E.g. $I(\Delta t) =
	\int_0^\infty \I(\omega) d\omega$.}
\begin{align}
	\I &= \langle |\J_x|^2\rangle + \langle|\J_y|^2\rangle &
	\Q &= \langle |\J_x|^2\rangle - \langle|\J_y|^2\rangle &
	\U &= 2\Re\langle \J_x\J_y^*\rangle &
	\V &= -2\Im\langle \J_x\J_y^*\rangle
\end{align}
so we need to evaluate $J_{xx} = \langle |\J_x|^2\rangle$,
$J_{yy} = \langle |\J_y|^2\rangle$ and $J_{xy} = \langle \J_x\J_y^*\rangle$.
For the left horn\footnote{The right horn follows by symmetry: $(L,A,B)\leftrightarrow
(R,B,A)$.} we get
\begin{align}
	J^L_{xx} &= \langle \J^L_x\J^{L*}_x \rangle \notag \\
		&= \frac12\big[1+\cos(\omega\Delta t)\big]\langle\J^A_x\J^{A*}_x\rangle
		+ \frac12\big[1-\cos(\omega\Delta t)\big]\langle\J^B_y\J^{B*}_y\rangle
		- \frac{i}2\langle\J^A_x\J^{B*}_y + \J^{A*}_x\J^B_y\rangle\sin(\omega\Delta t) \notag \\
		&= \frac14\big[\I^A+\I^B+\Q^A-\Q^B + (\I^A-\I^B+\Q^A+\Q^B)\cos(\omega\Delta t) -
		4\Im(\J^A_x\J^{B*}_y)\sin(\omega\Delta t) \big] \\
	J^L_{yy} &= \frac14\big[\I^B+\I^A+\Q^B-\Q^A - (\I^B-\I^A+\Q^B+\Q^A)\cos(\omega\Delta t) -
		4\Im\langle\J^B_x\J^{A*}_y\rangle\sin(\omega\Delta t) \big ]\\
	J^L_{xy} &= \frac14\big[\U^A-\U^B - i\V^A-i\V^B + (\U^A+\U^B -i\V^A+i\V^B)\cos(\omega\Delta t)
		+2i\langle\J^A_x\J^{B*}_x+\J^B_y\J^{A*}_y\rangle\sin(\omega\Delta t)
\end{align}
Hence
\begin{align}
	\I^L &= \frac12\big[\I^A+\I^B+(\I^A-\I^B)\cos(\omega\Delta t)
		-2\Im\langle\J^A_x\J^{B*}_y+\J^B_x\J^{A*}_y\rangle\sin(\omega\Delta t) \big] \notag \\
	\Q^L &= \frac12\big[\Q^A-\Q^B+(\Q^A+\Q^B)\cos(\omega\Delta t)
		-2\Im\langle\J^A_x\J^{B*}_y-\J^B_x\J^{A*}_y\rangle\sin(\omega\Delta t) \big] \notag \\
	\U^L &= \frac12\big[\U^A-\U^B+(\U^A+\U^B)\cos(\omega\Delta t)
		-2\Im\langle\J^A_x\J^{B*}_x+\J^B_y\J^{A*}_y\rangle\sin(\omega\Delta t) \big] \notag \\
	\V^L &= \frac12\big[\V^A+\V^B+(\V^A-\V^B)\cos(\omega\Delta t)
		-2\Re\langle\J^A_x\J^{B*}_x+\J^B_y\J^{A*}_y\rangle\sin(\omega\Delta t) \big]
\end{align}
The value of the barrel cross-terms depends on whether PIXIE is in single
or double barrel mode.

\paragraph{Single barrel mode}
In single barrel mode only one barrel is exposed to the sky; the other one
observes a static calibrator object. The light entering the two barrels is
therefore uncorrelated, and all the cross-terms disappear.
\begin{align}
	\I^L &= \frac12\big[\I^A+\I^B+(\I^A-\I^B)\cos(\omega\Delta t)\big] &
	\Q^L &= \frac12\big[\Q^A-\Q^B+(\Q^A+\Q^B)\cos(\omega\Delta t)\big] \notag \\
	\V^L &= \frac12\big[\V^A+\V^B+(\V^A-\V^B)\cos(\omega\Delta t)\big] &
	\U^L &= \frac12\big[\U^A-\U^B+(\U^A+\U^B)\cos(\omega\Delta t)\big]
\end{align}

\paragraph{Double barrel mode}
In double barrel mode the two barrels are both coaligned exposed to the sky,
so they observe the same wavefront entering. As PIXIE's angular resolution is
not infinite it is sensitive to wavefronts that are off-axis by a few degrees.
Light arriving from direction $\hat n$ will hit Barrel B a time
$\tau = \hat n \cdot \vec b / c$ before barrel A, where $\vec b$ is the distance vector from
barrel A to barrel B, and $c$ is the speed of light (see~\cite{pixie_array}).
So in this case $\vec \J^B = \gamma \vec\J^A$ with $\gamma = e^{-i\omega\tau}$.
\begin{align}
	\I^L &= \I^A+\V^A\cos(\omega\tau)\sin(\omega\Delta t) &
	\Q^L &= \Q^A\cos(\omega\Delta t)-\U^A\sin(\omega\tau)\sin(\omega\Delta t) \notag \\
	\V^L &= \V^A-I^A\cos(\omega\tau)\sin(\omega\Delta t) &
	\U^L &= \U^A\cos(\omega\Delta t)-\Q^A\sin(\omega\tau)\sin(\omega\Delta t)
\end{align}
If the barrels are not perfectly collimated, or if they have asymmetric sidelobes,
then the situation will be more complicated, as only part of the radiation that
enters the barrels will be correlated.

The leakage terms are not yet implemented in the simulator described in this
article. However, as they are all proportional to $\sin(\omega\Delta t)$ and
are therefore an antisymmetric function of the mirror stroke, they can be
easily disentangled from the main signal which is symmetric. Their absence
from the simulation should therefore impact our results meaninfully.

In the absence of the leakage terms, the signal in double barrel mode is
identical to that of single barrel mode with $(I^A,Q^A,U^A,V^A)=(I^B,Q^B,U^B,V^B)$.

\subsection{Detectors}
PIXIE has an x and y-oriented detector in each horn. The power deposited on each
of these is
\begin{align}
	P^L_x &= \frac14 \int_0^\infty \big(\I^L(\omega)+\Q^L(\omega)\big) d\omega &
	P^L_y &= \frac14 \int_0^\infty \big(\I^L(\omega)-\Q^L(\omega)\big) d\omega
\end{align}
where all the quantities depend on $\Delta t$ and potentially $\tau$. Here
one factor of 1/2 comes from the linear combination of $I$ and $Q$, while
the other one takes into account that the power of a harmonic wave is half
its squared amplitude.

\section{What PIXIE observes}
Nominally, what PIXIE's 4 detectors observe at any given time is
\begin{align}
\vec s_\textrm{det}(t) &= \overbrace{\frac14\begin{bmatrix}
	\vec e_I+\vec e_Q & 0 \\
	\vec e_I-\vec e_Q & 0 \\
	0 & \vec e_I+\vec e_Q  \\
	0 & \vec e_I-\vec e_Q \end{bmatrix}}^\textrm{Detector response}
	\cdot
	\overbrace{\frac12\begin{bmatrix}
	1 &  D_I-D_P & 1 & -D_I+D_P \\
	D_I-D_P	& 1 & -D_I+D_P & 1
	\end{bmatrix}}^\textrm{Horn response}
	\cdot
	\overbrace{\begin{bmatrix}
	R_{1t}\cdot\vec s_{\textrm{sky}1}(\hat p_{1t},0) \\
	R_{2t}\cdot\vec s_{\textrm{sky}2}(\hat p_{2t},0) \\
	R_{1t}\cdot\vec s_{\textrm{sky}1}(\hat p_{1t},\tau) \\
	R_{2t}\cdot\vec s_{\textrm{sky}2}(\hat p_{2t},\tau)
	\end{bmatrix}}^\textrm{Barrel signal}
	\label{eq:response}
\end{align}
where $\hat p_{bt}$ is the sky pointing of barrel $b$ at time $t$,
$\vec s_{\textrm{sky},b}(\hat p,\tau)$ is the beam-smoothed,
frequency-weighted sky autocorrelation
function for the given pointing and time delay as seen by barrel $b$
(different barrels can see different skies because one barrel may
be covered by a blackbody calibrator),
$R_{bt}$ is
a matrix that rotates the polarization basis from sky to instrument
coordinates, $D_I$ and $D_P$ are Stokes mixing matrixes given by
$D_I = \textrm{diag}(1,0,0)$ and $D_P = \textrm{diag}(0,1,1)$,
and $\vec e_I = (1,0,0)$ and $\vec e_P = (0,1,1)$ are Stokes basis
vectors. PIXIE's interferometry shows up in two ways here:
The sky autocorrelation function, rather than just its intensity,
is what is measured; and the barrel signal differencing in the horn
response.

Of course, a real instrument does not read out data with infinite
time resolution, but as a set of discrete samples, each of which is
noisy. The PIXIE hardware will also apply a bandpass filter to avoid
aliasing and suppress low-frequency noise. Taking this into account,
we model the time-ordered data as
\begin{align}
	\vec d_i = B_{ij} \int_{t_j-\Delta t/2}^{t_j+\Delta t/2} \vec s_\textrm{det}(t) \textrm{d}t + \vec n_i
\end{align}
where $i$ is the sample index, $B$ is the bandpass filter, $\Delta t$ is the sample
interval and $\vec n_i$ is the noise in sample $i$.

To simulate this, we need to be able to
\begin{enumerate}
	\item Generate a simulated sky
	\item Compute the telescope pointing $\hat p$ and associated polarization rotation
	matrix $R$ and relative path delay $\tau$ at each time $t$ for each barrel.
	\item Evaluate the autocorrelation function for each component of the sky as
	seen by each barrel for any $\hat p$ and $\tau$.
	\item Find the detecor response using equation~\ref{eq:response}.
	\item Integrate across subsample variation, and apply the time-frequency
	response and detector noise.
\end{enumerate}

\section{Simulating the sky components}

\section{Pointing}
PIXIE will orbit at the Sun-Earth L2 point, placing it in the ecliptic, with
a heliocentric ecliptic latitude $b=0$ and longitude $l=l_0 +
360\degree\frac{t-t_0}{T_\textrm{orbit}}$ with $T_\textrm{orbit} = 1\textrm{year}$.
As it orbits scans great circles
perpendicular to the direction towards the sun, with a linearly increasing scan angle
$\alpha_\textrm{scan} = \alpha_{\textrm{scan},0} + 360\degree \frac{t-t_0}{T_\textrm{scan}}$.
To form an actual great circle rather than an epicycle, the scan axis does not
move continuously with $b$, but updates in steps after each circle has been completed:
$\alpha_\textrm{orbit} = l_0 + 360\degree\left\lfloor\frac{t-t_0}{T_\textrm{scan}}\right\rfloor\frac{T_\textrm{scan}}{T_\textrm{orbit}}$.
During each scan the the telescope spins around around its boresight to modulate
the observed polarization and reject systematics: $\alpha_\textrm{spin} =
\alpha_{\textrm{spin},0} + 360\degree\frac{t-t_0}{T_\textrm{spin}}$. And finally,
while it is spinning the dihedral mirror sweeps backwards and forwards at constant
speed, varying the path length time difference in the fourier transform spectrometer
by $\tau = A_\textrm{delay}c^{-1} \textrm{triangle}\Big(\frac{t-t_0}{T_\textrm{stroke}}$,
with $A_\textrm{delay} = 10.40303 \textrm{mm}$ for the purposes of this paper, but
varying somewhat by observing mode in the real experiment, and with
$\textrm{triangle}(x)$ being the triangle wave with period 1, mean 0 and a zero
crossing at $x=0$.

To allow PIXIE mapmaking to use fast fourier transform methods, the stroke, spin
and scan periods will be synchronized such that there is an integer number of
strokes in a spin, and an integer number of spins in a scan. We will use the
values $T_\textrm{spin} = 60 \textrm{s}$, $T_\textrm{stroke} = T_\textrm{spin}/8 =
7.5 \textrm{s}$, $T_\textrm{scan} = 384 T_\textrm{spin} = 384 \textrm{min}$ here.
However, the simulation purposefully does not depend on integer ratios to be able
to investigate the consequences of small deviations from integer ratios.

A barrel-to-sky rotation matrix that implements this pointing model is
\begin{align}
	R_\textrm{tot}(b,t)  &= R_\textrm{orient}(t)R_\textrm{barrel}(b) \\
	R_\textrm{orient}(t) &= R_z\big(\alpha_\textrm{orbit}(t)\big)
		R_y\Big(\frac\pi2-\alpha_\textrm{eclip}\Big)
		R_z\big(\alpha_\textrm{scan}(t)\big)R_y\Big(\frac\pi2-\alpha_\textrm{open}\Big)
		R_z\big(\alpha_\textrm{spin}(t)\big) \\
	R_\textrm{barrel}(b) &= R_z\big(\Delta\phi(b)\big)R_y\big(\Delta\theta(b)\big)R_z\big(\Delta\psi(b)\big)
\end{align}

Here $R_\textrm{barrel}(b)$ represents the orientation of barrel $b$ relative
to the spacecraft. Fiducially $R_\textrm{barrel} = 1$ for both barrels, but
we include this rotation to be able to support misaligned barrels or more
complicated beams. $R_\textrm{orient}(t)$ represents PIXIE's orientation
in space at time $t$, and in addition to the angles described above includes
$\alpha_\textrm{eclip}$ and $\alpha_\textrm{open}$, which represent the
offset of PIXIE's orbital plane from the ecliptic and the opening angle
offset (to support non-great-circle scans), both of which are fiducially 0.
$R_y(\theta)$ and $R_z(\theta)$ are rotations around the $y$ and $z$ axes
by an angle $\theta$.

$R_\textrm{tot}$ encodes both the sky coordinates and polarization rotation.
\begin{align}
	x_i &= R_\textrm{tot,xi} & y_i &= R_\textrm{tot,yi} & p_i &\equiv z_i = R_\textrm{tot,zi} \\
	l        &= \tan^{-1}(p_y/p_x)  &
	b        &= \tan^{-1}\Big(\frac{p_z}{\sqrt{p_x^2+p_y^2}}\Big) &
	\gamma   &= \tan^{-1}\Big(\frac{x_z}{p_y x_x - p_x x_y}\Big)
\end{align}
with all the above being a function of the barrel index and time.
$\hat p = (p_x,p_y,p_z)$ is the pointing vector and $\gamma$ is the
polarization basis rotation, and corresponds to a Stokes rotation
matrix
\begin{align}
	R &= \begin{bmatrix}
		1 & 0 & 0 \\
		0 & \cos(2\gamma) & -\sin(2\gamma) \\
		0 & \sin(2\gamma) & \cos(2\gamma)
	\end{bmatrix}
\end{align}

%	p_{bt,i} &= R_{\textrm{tot,3i}}(b,t) \\

\section{Evaluating the sky autocorrelation function at the observed location}
As PIXIE observes the sky it mesures the autocorrelation function of the
radiation coming from the points it scans past. To simulate the PIXIE signal
we therefore need to be able to evaluate the I, Q and U autocorrelation functions at
an arbitrary point $\hat p$ on the sky for an arbitrary phase delay $\tau$
for each component that makes up the sky (CMB, dust, etc.).

\subsection{Approaches that don't work}
A straightforward and general way of doing this would be to precompute the
full-sky autocorrelation function: Evaluate the full-sky spectrum at equi-spaced
frequencies. Apply the beam to each frequency map and scale each frequency by
the instrument's frequency response. Fourier transform the result
to get a (pixelized version of) the full-sky autocorrelation function. To read off
the value at a general $(\hat p, \tau)$ one would then do an interpolated lookup
in this $N_\textrm{pix}$ by $N_\textrm{delay}$ data cube. This approach has the
advantage of being able to handle frequency-dependent beams, which is otherwise
very hard to implement. However, in order to be able to investigate the effect
of sub-resolution features (both spectrally and spatially) this data cube would
need to be pixelized at many times higher resolution than the PIXIE output map.
This made the memory requirements of this approach prohibitively high. For example,
for $0.1\degree$ spatial resolution and 5000 frequency bins, storing the full-sky
autocorrelation function would need about 700 GB of RAM.

If one assumes a frequency-independent beam, which should be a good approximation
for PIXIE, then it's sufficient to apply the beam to the parameters used to generate
the spectrum (e.g. the CMB temperature map) rather than the spectrum itself. This
decouples the spatial and spectral dimensions, making it possible to evaluate the
spectrum in one pixel independently of the rest of the sky. With this, we could
imagine the following approach: For each sample, interpolate the spectrum parameters
at $\hat p$, then evaluate the whole spectrum, apply the frequency response,
fourier transform it,
and interpolate the value for $\tau$. In our example above, this would reduce
the RAM requirements by a factor of 5000. But it would introduce another prohibitive
cost: The need to evaluate the spectrum at thousands of frequencies and fourier transform
these for every sample in the TOD.\footnote{
A hybrid approach between these two would be to precompute the autocorrelation function
for a chunk of the sky around the current sample, and reuse that for subsequent samples
until a sample falls outside the chunk, and then precompute a new chunk. We investigated
this in the hopes of being able to support frequency-dependent beams, but found that
edge effects and the flat-sky-approximation needed to perform beam-smoothing on a small
patch did not result in the required accuracy. This may still be a good approach for
frequency-independent beam simulations, though.}

\subsection{Autocorrelation by Taylor expansion}
In the end, we went for a Taylor expansion approach: The autocorrelation function
is evaluated as a perturbation around a different but similar
precomputed autocorrelation function. This is done differently for each sky component.

\subsubsection{CMB}
Taking into account the instrument's frequency response $\rho(\nu)$, PIXIE
observes the CMB with the spectrum
\begin{align}
I^\textrm{CMB}_{\nu,I}(\hat p, \nu) &= \rho(\nu)B_\nu(\nu, T(\hat p)) = \frac{2h\nu^3\rho(\nu)}{c^2}\frac{1}{e^{\frac{h\nu}{kT(\hat p)}} - 1}
\end{align}
Here the $I$ subscript indicates the Stokes parameter, and $T(\hat p)$ is the
\emph{beam-smoothed}\footnote{Implementation of asymetric beams is described in
section X.} CMB temperature at pointing $\hat p$.
Including the doppler dipole, T only has a contrast of order $10^{-3}$, so a Taylor
expansion in T will converge rapidly. Our goal is $<10^{-9}$ relative error, so an
expansion to 3rd order, which should give order $10^{-12}$ error, should be sufficient.
The expansion is
\begin{align}
	I^\textrm{CMB}_{\nu,I}(\hat p, \nu) &= f_0(\nu) + f_1(\nu) \Delta T + \frac12 f_2(\nu) \Delta T^2 + \frac16 f_3(\nu) \Delta T^3
\end{align}
where
\begin{align*}
	f_0 &= p(g_0-1)^{-1}    & g_0 &= e^{a/T_0} \\
	f_1 &= pf_0^2 g_1       & g_1 &= -g_0 a/T_0^2 \\
	f_2 &= p(-2 f_0 f_1 g_1 - f_0^2 g_2) &
	g_2 &= a(2 g_0-T_0 g_1)/T_0^3 \\
	f_3 &= p(-2 f_1^2 g_1 - 2f_0 f_2 g_1 - 4 f_0 f_1 g_2 - f_0^2 g_3) &
	g_3 &= -(3/T_0 + a/T_0^2)g_2 + a g_1/T_0^3
\end{align*}
and where $p = \frac{2h\nu^3\rho(\nu)}{c^2}$, $a = h\nu/k$ and $\Delta T(\hat p) = T(\hat p)-T_0$ with $T_0 = 2.725$K. 

The autocorrelation function is simply the fourier transform of the spectral power
density,
\begin{align}
I_\tau(\tau) &= \int_{-\infty}^\infty e^{2\pi i \nu \tau} I_\nu(\nu) d\nu = \tilde I_\nu(\tau)
\end{align}
Applying this to the Taylor expansion, we get
\begin{align}
I^\textrm{CMB}_{\tau,I}(\hat p,\tau) &= \tilde f_0(\tau) + \tilde f_1(\tau) \Delta T + \frac12 \tilde f_2(\tau) \Delta T^2 + \frac16 \tilde f_3(\tau) \Delta T^3
\end{align}
Hence, we can compute the autocorrelation for any $\hat p,\tau$ if we simply precompute
the four position-independent functions $\{f_i\}$.\footnote{
If we had not needed to support the frequency response of the instrument,
we could have avoided the Taylor expansion by absorbing variation in $T$
into rescaling of $v$. Sadly, PIXIE has significant damping at high frequency,
so this approach does not work.} The CMB has frequency-independent polarization,
so the $Q,U$ autocorrelation functions can be derived from $I$ by scaling them by
the local $Q,U$ polarization fractions. I.e. $I^\textrm{CMB}_{\tau,Q|U}(\hat p,\tau) = I^\textrm{CMB}_{\tau,I}(\hat p,\tau) \frac{I^\textrm{CMB}_{\textrm{ref},Q|U}(\hat p)}{I^\textrm{CMB}_{\textrm{ref},I}(\hat p)}$.

\subsubsection{Dust}
We model the dust as a modified blackbody with constant
$T=19.6$K and $\beta=1.59$, but varying opacity. The observed spectrum is thus
\begin{align}
I^\textrm{dust}_{\nu,i}(\hat p,\nu) &= A_i(\hat p) \frac{h\nu^{3+\beta}\rho(\nu)}{c^2}\frac{1}{e^{\frac{h\nu}{kT}}-1} \equiv A_i(\hat p) f_{0\beta}(\nu)
\end{align}
for $i \in \{I,Q,U\}$. Here the prefactor $A_i(\hat p)$ encodes the position-dependent
dust opacity and polarization, and should be smoothed with the instrument beam.
Since $T$ and $\beta$ are constant, the frequency-dependent part of this spectrum is
already position-independent, so we don't actually need to Taylor-expand in this case.
We just need to precompute a single autocorrelation shape which is rescaled for each
pointing.
\begin{align}
I^\textrm{dust}_{\tau,i}(\hat p,\tau) &= A_i(\hat p) \tilde f_{0\beta}(\tau)
\end{align}
This will need to be modified for more complicated dust models. If $T$ or $\beta$
only change slightly, then the Taylor expansion approach can be used. For more
substantial variation, a better approach may be to model it as several dust components,
each with fixed parameters.

\subsubsection{Other components}
The results reported here are based on simulations that only include CMB and dust,
but other components such as synchrotron, free-free, CO, AME, etc. can be implemented
in a similar vein as above, as long as they can be approximated as a sum of
constant-spectral-shape components or can be Taylor-expanded to sufficient accuracy.

\section{TOD simulation}
\emph{This was written independently of the above, and needs to be unified with it.
This section is still very incomplete}

PIXIE observs the sky with two collimated
barrels. There are two modes of observation: double barrel observation,
where both barrels see the same sky; and single barrel observation, where
one barrel is covered by a blackbody calibrator. In both cases, the
radiation entering the barrels passes through a fourier transform
spectometer before entering the detector feedhorns. Aside from the resolution
effects of the telescope beam, the response of this assembly is
\begin{align}
	I^A_\textrm{horn} &= \frac12 [I^A+I^B]_0 + \frac12[I^A-I^B]_\tau \\
	Q^A_\textrm{horn} &= \frac12 [Q^A-Q^B]_0 + \frac12[Q^A+Q^B]_\tau \\
	U^A_\textrm{horn} &= \frac12 [U^A-U^B]_0 + \frac12[U^A+U^B]_\tau,
\end{align}
The other horn follows by symmetry.
Here $[\ldots]_\tau$ means that
the bracketed quantities were evaluated with a time delay $\tau$. For
example $[I]_\tau \equiv \langle E_x(t)E_x(t+\tau)\rangle + \langle E_y(t)E_y(t+\tau)\rangle$.
These $\tau$ terms represent the autocorrelation function of the incoming
light, which is related to the spectrum of the light through a fourier transform.
The zero-delay (DC) terms are for the most part irrelvant. Because they are not
modulated by the mirror motion they change so slowly that they are indistinguishable
from long-term correlated noise in the detectors. Nevertheless, they are included
in the simulation.

PIXIE has two detectors in each horn. They are fiducially sensitive to orthogonal
polarizations of light, with one measuring power of $E_x$ (corresponding to a Stokes
response of $\frac12(I+Q)$) and the other $E_y$ (corresponding to $\frac12(I-Q)$).
This orthogonality is useful for rejecting systematic errors, but we allow an arbitrary
Stokes response per detector in the simulator to be able to test the effect of any
non-orthoginality.

\end{document}
